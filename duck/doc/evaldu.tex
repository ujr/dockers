% Dokumentation zu "evaldu" (disk usage evaluation)
% ujr/2008-02-26 for xilab

\magnification=\magstep1
\font\titlefont=cmbx12 at 16pt
\font\sectionfont=cmbx12
\def\section#1\par{\bigbreak
  \leftline{\sectionfont #1}%
  \nobreak\medskip\noindent}

% Verbatim listing...
\chardef\other=12
\def\ttverbatim{\begingroup\def\do##1{\catcode`##1=\other}%
 \dospecials \catcode`\|=\other \obeyspaces \obeylines \tt}
{\obeyspaces\gdef {\ }} % \obeyspaces now gives \ , not \space

\catcode`\|=\active
\newskip\ttglue \ttglue=.5em plus.25em minus.15em
{\obeylines\gdef|{\ttverbatim\spaceskip=\ttglue%
 \let^^M=\ \let|=\endgroup}}

\font\ninett=cmtt9
\def\begincode{\medskip
 \ttverbatim \let\tt=\ninett
 \baselineskip=10pt plus .2pt minus .2pt
 \catcode`\|=\other \catcode`\\=0
 \def\\{\char"5C\relax}% backslashes must be escaped
 \let\par=\endgraf \rightskip=-5pt}
\def\endcode{\par\endgroup\medbreak\noindent}


\topskip 4pc
\centerline{Urs-\kern-.2ex Jakob R\"uetschi, Xilab}
\bigskip
\centerline{\titlefont Disk Usage Accounting}
\smallskip
\centerline{\bf A Simple Solution for Linux Systems}
\vskip 2pc

\noindent Xilab bietet gegen Entgelt Speicherplatz im Internet an.
Das Entgelt des Kunden richtet sich nach dem tats\"achlich belegten
Speicherplatz auf den Xilab-Servern. Die Belegung muss daher f\"ur
jeden Kunden regelm\"assig erhoben, periodisch konsolidiert und in
Rechnung gestellt werden.

\section Funktionsweise

Die Disk Usage jedes Kunden wird t\"aglich erhoben und in einem Logfile
abgelegt. Die Speicherbelegung wird dabei auf ganze MegaBytes gerundet.
Bei der Rechnungstellung wird der durchschnittliche Speicherbedarf
pro Monat ausgewiesen und verrechnet. Der Durchschnitt wird mit der
exakten Anzahl Tage (28, 29, 30 oder 31) des Monates berechnet.

Die Erhebung des belegten Speicherplatzes erfolgt automatisch
durch den Server. Der Durchschnittswert pro Monat wird durch
ein Skript vorgenommen, welches manuell gestartet wird. Sein
Output wird per Copy/Paste in die Rechnung (oder in ein Excel Sheet)
eingef\"ugt.

\section Messung der Disk Usage

Idealerweise wird ein File System mit Quota-Support verwendet, weil
dieses die Buch\-f\"uhrung der durch einen User und/oder eine Gruppe
belegten Speicherplatzes automatisch vornimmt.
%
Steht kein solches File System zur Verf\"ugung, kann das Standard
Unix Tool {\bf du}(1) verwendet werden, welches von einem oder
mehreren Startpunkten aus rekursiv durch den Verzeichnisbaum geht
und den belegten Disk Space aufsummiert.

In beiden F\"allen ist der t\"agliche Wert zusammen mit dem
Datum in ein Logfile zu schreiben. Bei der Verwendung von
{\bf du}(1) ist zu bedenken, dass das eine ``teure'' Operation
ist und entsprechend viel Zeit beanspruchen kann.

\section Implementation

Das Shell Skript {\tt log-disk-usage.sh} wird jede Nacht
und f\"ur jedes Kundenverzeichnis {\it /pfad/kunde\/} durch
den {\bf cron}-Dienst aufgerufen wird. Das Kundenverzeichnis
wird dem Skript als Parameter \"ubergeben. Die gefundene
Disk Usage wird zusammen mit einer Datumsangabe an die Datei
{\it /pfad/kunde\tt.du} angeh\"angt.
\smallskip
\item{} Skriptaufruf: {\tt /root/bin/log-disk-usage.sh} {\it /pfad/kunde}
\item{} Speicherort des Logfiles: {\it /pfad/kunde\tt.du}
\item{} Zeilenformat im Logfile: {\it yyyy mm dd Mbytes}
\smallskip
\noindent Die {\tt/\it dirname\tt.du}-Dateien wachsen best\"andig
und werden (gegenw\"artig) nicht rotiert.

Vor der Rechnungstellung werden die {\tt .du}-Dateien
mit dem Skript {\tt eval-du-log.awk} konsolidiert, d.h., die
t\"aglichen Werte zu einem Wert pro Monat aggregiert.
\smallskip
\item{} Skriptaufruf: {\tt /root/bin/eval-du-log.awk}
 [{\tt free=$N$}] [{\tt cost=$K$}] {\it /pfad/kunde\tt.du}
\smallskip
\noindent Dabei ist {\tt free=$N$} die Anzahl MBytes, die im
Grundpreis eingeschlossen sind (default: $N=0$) und {\tt cost=$K$}
der Preis pro MByte (default: $K=1$).
Der Output ist eine Zeile pro Monat im Format
``{\it yyyy Monat Mbytes Cost},'' wobei {\it Mbytes\/} auf drei Stellen
nach dem Komma und {\it Cost\/} auf zwei Stellen ausgegeben wird.
Beispiel: 
\begincode
2007 Dezember 58.800 58.80
2008 Januar 120.000 120.00
2008 Februar 570.167 570.17
\endcode
Dieser Output kann per Copy/Paste in eine Tabelle oder direkt
in das Rechnungsdokument bef\"ordert werden.

\section Das Logging-Skript

Das Logging ist als einfaches Shell-Skript {\tt log-disk-usage.sh}
realisiert:
\begincode
#!/bin/sh
\smallskip
# Usage: log-disk-usage.sh <dir>
# To be called regularly by cron.
\smallskip
target="$1" || exit 127
log="${target%/}.du"
today=`date '+%Y %m %d'` || exit 1
set -- `du -s "$target"` && blocks=$1 || exit 1
echo $today $(((blocks+512)/1024)) >> "$log"
\endcode
Auf der letzten Zeile wird die Anzahl Disk Blocks (Linux: 1024 Bytes)
von {\bf du}(1) auf MBytes gerundet.
Hier ein Beispiel des generierten Logfiles:
\begincode
\indent\vdots
2008 01 28 68
2008 01 29 73
2008 01 30 72
2008 01 31 72
2008 02 01 123
2008 02 02 125
2008 02 03 125
2008 02 04 144
\indent\vdots
\endcode

\section Das Konsolidierungs-Skript

Die Konsolidierung zur durchschnittlichen monatlichen Belegung
und die Bestimmung des Preises daf\"ur erledigt das AWK-Skript
{\tt eval-du-log.awk}:
\begincode
#!/usr/bin/awk -f
\smallskip
# Evaluate a <year> <month> <mday> <xbytes> disk usage log.
# Usage: awk -f eval-du-log [free=N] [cost=K] <logfile>
\medskip
BEGIN { if (!cost) cost = 1
  mon["01"] = "Januar"; mon["02"] = "Februar"; mon["03"] = "Maerz"
  mon["04"] = "April"; mon["05"] = "Mai"; mon["06"] = "Juni"
  mon["07"] = "Juli"; mon["08"] = "August"; mon["09"] = "September"
  mon[10] = "Oktober"; mon[11] = "November"; mon[12] = "Dezember"
  print "#year month load cost"
}
\medskip
{ mbytes = $4
  year = $1; month = $2
  tot[year,month] += mbytes
  cnt[year,month] += 1
}
\medskip
END { for (s in tot) A[++n] = s
  isort(A, n) # sort by year then month
  for (i = 1; i <= n; i++) {
    split(A[i], a, SUBSEP)
    y = a[1]; m = a[2]
    d = tot[y,m]/cnt[y,m]
    c = (d - free) * cost
    printf "%s %s %.03f %.02f\\n", y, mon[m], d, c
  }
}
\medskip
# Insertion sort of A[1..n] (from AWK man page)
function isort(A, n,   i, j, hold)
{
  for (i = 2; i <= n; i++)
  {
    hold = A[j=i]
    while (A[j-1] > hold)
    { j--; A[j+1] = A[j] }
    A[j] = hold
  }
  # sentinel A[0] = "" will be created if needed
}
\endcode
Im BEGIN-Block wird der Default-Wert f\"ur die {\it cost}-Variable
explizit gesetzt (weil nicht~0) und ein Array aufgebaut, der von
Monatsnummern (mit f\"uhrender Null!) nach Monatsnamen \"ubersetzt.

Der n\"achste Block wird f\"ur jede Zeile des Inputs ausgef\"uhrt.
Er aktualisiert Totalwert und Anzahl in zwei assoziativen Arrays,
die mit Jahr und Monat (mit f\"uhrender Null!) indiziert werden.

Der END-Block kopiert den assoziativen Array {\it tot\/}
in den Integer-indizierten Array~$A$, welcher dann sortiert und
schliesslich ausgegeben wird. Die Sortier-Routine wurde der
Manual Page zu {\bf mawk} entnommen.

\section Hinweise

Wenn in einem Logfile Eintr\"age fehlen weil das Logging-Skript
nicht ausgef\"uhrt wurde (z.B. weil der Rechner nicht lief), dann
wird bloss \"uber die vorhandenen Eintr\"age gemittelt, was immer
noch zu einem g\"ultigen Resultat f\"uhrt.
\medskip
\noindent
Hier vorgestellt wurde eine einfache L\"osung f\"ur ein einfaches
Problem. Sollte das Problem gr\"osser werden (mehr Kunden!), muss
diese Abrechnungsl\"osung \"uberdacht und angepasst werden.

\nopagenumbers\vskip 1pc plus 1filll
\centerline{Created 2008-02-26 by Urs-\kern-.2ex Jakob R\"uetschi}
\centerline{Copyright 2008 by Xilab. All rights reserved.}
\bye
